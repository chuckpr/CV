% LaTeX Curriculum Vitae Template
%
% Copyright (C) 2004-2009 Jason Blevins <jrblevin@sdf.lonestar.org>
% http://jblevins.org/projects/cv-template/
%
% You may use use this document as a template to create your own CV
% and you may redistribute the source code freely. No attribution is
% required in any resulting documents. I do ask that you please leave
% this notice and the above URL in the source code if you choose to
% redistribute this file.

\documentclass[letterpaper]{article}

\usepackage{hyperref}
\usepackage{geometry}

% Comment the following lines to use the default Computer Modern font
% instead of the Palatino font provided by the mathpazo package.
% Remove the 'osf' bit if you don't like the old style figures.
\usepackage[T1]{fontenc}
\usepackage[sc,osf]{mathpazo}

% Set your name here
\def\name{\textbf{Charles Pepe-Ranney}}

% Replace this with a link to your CV if you like, or set it empty
% (as in \def\footerlink{}) to remove the link in the footer:
\def\footerlink{}

% The following metadata will show up in the PDF properties
\hypersetup{
  colorlinks = true,
  urlcolor = blue,
  pdfauthor = {\name},
  pdfkeywords = {},
  pdftitle = {\name: Curriculum Vitae},
  pdfsubject = {Curriculum Vitae},
  pdfpagemode = UseNone
}

\geometry{
  body={7.0in, 9.5in},
  left=0.75in,
  top=0.60in,
  bottom=0.60in
}

% Customize page headers
\pagestyle{myheadings}
\markright{\name}
\thispagestyle{empty}

% Custom section fonts
\usepackage{sectsty}
\sectionfont{\rmfamily\mdseries\itshape\large}
\subsectionfont{\rmfamily\mdseries\itshape\normalsize}

% Other possible font commands include:
% \ttfamily for teletype,
% \sffamily for sans serif,
% \bfseries for bold,
% \scshape for small caps,
% \normalsize, \large, \Large, \LARGE sizes.

% Don't indent paragraphs.
\setlength\parindent{0em}

% Make lists without bullets
\renewenvironment{itemize}{
  \begin{list}{}{
    \setlength{\leftmargin}{1.5em}
  }
}{
  \end{list}
}

% include multibib package
%\usepackage{multibib}

\begin{document}

% Place name at left
{\Large \name}

% Alternatively, print name centered and bold:
%\centerline{\large \bf \name}

\vspace{0.125in}

\begin{minipage}{0.45\linewidth}
  Cornell University \\
  Department of Crop and Soil Sciences \\
  \href{http://www.css.cornell.edu/faculty/buckley/}{Buckley Lab} \\
  Ithaca, NY 14850
\end{minipage}
\begin{minipage}{0.45\linewidth}
  \begin{tabular}{ll}
    Phone: & (575) 313-0993 \\
    Email: & \href{mailto:chuck.peperanney@gmail.com}{\tt chuck.peperanney@gmail.com} \\
  \end{tabular}
\end{minipage}
\section*{Professional Preparation}
\begin{itemize}
    \item \textbf{B.S. Engineering (high honors)} - Environmental Science Specialty, Colorado School of Mines 2006.
    \item \textbf{M.S. Environmental Engineering} - Biotechnology and Environmental Microbiology Emphasis, Colorado School of Mines 2009.
    \item \textbf{PhD Environmental Science and Engineering Division}, Colorado School of Mines 2012.
\end{itemize}
\section*{Relevant Experience}
\begin{itemize}
        {\small
    \item Proficient with \textbf{Python}, \textbf{R}, Latex, Bash
        scripting and \textbf{Linux system administration}. Experience
        with Perl, javascript, PostgreSQL, MySQL
    \item Thorough understanding and fluent with many data
        science/bioinformatics libraries including IPython notebooks,
        GGPlot2 (R), Matplotlib (Python), Pandas (Python), phyloseq
        (R), plyR/dplyR/tidyR (R) , QIIME, Mothur, Khmer, and BioPython.
        Experience with Bokeh (Python), D3.js (my bl.ocks:
        \href{http://bl.ocks.org/chuckpr}{bl.ocks.org/chuckpr}), ggvis
        (R) and lattice (R).
    \item Currently taking online courses for the \textbf{Data Science
        Signature Track} with Coursera -- Passed and received verified
        certificate with distinction for 
        \href{https://www.coursera.org/verify/TVZ2AX26SJ}{The Data Scientist's Toolbox},
        \href{https://www.coursera.org/verify/SB25Y92UJ6}{R Programming}, 
        \href{https://www.coursera.org/account/accomplishments/records/J75Um8uqQcrzBGDg}{Exploratory Data Analysis}, 
        \href{https://www.coursera.org/account/accomplishments/records/TXAWtJZV6hSe2U8G}{Statistical Inference}, and
        \href{https://www.coursera.org/account/accomplishments/records/pTVzUYHgx4sbcUmY}{Regression Models}
        courses.
    \item Experience developing amplicon sequencing protocols from the
        ground up for next-generation-sequencing technologies (454 and
        Illumina) with SSU rRNA genes and Fungal ITS amplicons.
%    \item Taught laboratory and bioinformatics methods for microbial
%        community analysis at the Microbial Diversity Course (Marine
%        Biology Laboratory, Woods Hole, MA).  
        }
\end{itemize}
%\section*{Synergistic Activities}
%\begin{itemize}
%    \item \emph{Teaching fellow, Microbial Diversity Course.}  The summers
%        of 2010, 2011 and 2012 I taught bioinformatics methods and
%        molecular laboratory techniques as a Microbial Diversity Course
%        teaching fellow (Marine Biology Laboratory, Woods Hole, MA).
%        Additionally, I set-up and installed all computational analysis
%        software on the Linux machines and served as the system
%        administrator.  I will be returning to the course in the same
%        capacity for the 2013 summer.  Students are eager to learn a
%        variety of computational methods and possess a wide range of
%        prior computational experience.  Students leave with basic
%        foundational skills to manage the flood of biological sequence
%        information typical of contemporary microbiology investigations.
%\end{itemize}

\section*{Appointments}
\begin{itemize}
    \item \textbf{Research Assistant}, Environmental Science and Engineering
        Division, Colorado School of Mines (2006-2012)
    \item \textbf{Postdoctoral Researcher}, Laboratory of Daniel H Buckley, Department of Crop and Soil
        Sciences, Cornell University (2013-present)
    \item \textbf{Teaching Fellow}, Marine Biology Laboratory (Microbial
        Diversity Course) (2010-2014)
\end{itemize}
\section*{Awards and Fellowships}
\begin{itemize}
    \item \textbf{2010, 2011, 2012 and 2013 Teaching fellow for the Microbial
        Diversity Course} at the Marine Biological Laboratory, Woods
        Hole.  Course Directors: Daniel Buckley and Steve Zinder.
    \item \textbf{2006 Outstanding Graduating Senior Award}, Colorado School of Mines - Envirnomental Science and Engineering Division
    \item \textbf{2005 and 2006 Department of Energy Science Undergraduate Laboratory Internship
        (SULI)} at Idaho National Lab
    \item \textbf{2006 INL Undergraduate Scholarship}
\end{itemize}
\section*{Publications in Refereed Journals}
\begin{itemize}
\item Pepe-Ranney C, Berelson WM, Corsetti FA, Treants M, Spear JR.
    \textbf{Cyanobacterial construction of hot spring siliceous stromatolites in
    Yellowstone National Park, Wyoming}, 2012, {\it Environmental
    Microbiology} 14(5), 1182-1197. \href{http://www.ncbi.nlm.nih.gov/pubmed/22356555}{link} 
\item Berelson WM, Corsetti FA, Pepe-Ranney C, Hammond DE, Beaumont W, Spear
    JR. \textbf{Hot spring siliceous stromatolites in Yellowstone National Park:
    assessing growth rates and laminae formation}, 2011, {\it Geobiology}
    9(5), 411-424. \href{http://www.ncbi.nlm.nih.gov/pubmed/21777367}{link}
\item Osburn MR, Sessions AL, Pepe-Ranney C, Spear JR. \textbf{Hydrogen-isotopic
    variability in fatty acids from Yellowstone National Park hot spring
    microbial communities}, 2011, {\it Geochimica et Cosmochimica Acta}
    75(17), 4830-4845.
    \href{http://www.sciencedirect.com/science/article/pii/S0016703711003152}{link}
\item Br\"{a}uer S, Vuono D, Carmichael M, Pepe-Ranney C, Strom A, Rabinowitz
    E, Buckley DH, Zinder S. \textbf{Microbial sequencing analyses suggest
    the presence of a fecal veneer on indoor climbing wall holds.}, 2014, {\it
    Current Microbiology} \href{http://www.ncbi.nlm.nih.gov/pubmed/24972665}{link}
\item Pepe-Ranney C, Koechli C, Potrafka R, Garcia-Pichel F, Andam C, Eggleston E, Buckley DH. 
    \textbf{Non-cyanobacterial diazotrophs mediate dinitrogen fixation in
        biological soil crusts during early crust formation.}\\ 
    Accepted by at \textit{ISMEJ}, May 2015. Preprint:\\ 
    \href{http://dx.doi.org/10.1101/013813}
    {http://dx.doi.org/10.1101/013813}\\
    Code for sequence analysis and manuscript figures can be found here:\\
    \href{http://www.github.com/chuckpr/NSIP_data_analysis}
    {github.com/chuckpr/NSIP\_data\_analysis}
\end{itemize}
\section*{Submitted Journal Articles}
\begin{itemize}
    \item Pepe-Ranney C and Hall EK. \textbf{The effect of carbon subsidies on
        planktonic niche partitioning and recruitment during biofilm
        assembly.}\\
        In review at \textit{Frontiers of Aquatic Microbiology}. Preprint: \\
        \href{http://dx.doi.org/10.1101/013938}
        {http://dx.doi.org/10.1101/013938}\\
        Manuscript figures and corresponding code can be found here: \\
        \href{http://www.github.com/chuckpr/BvP_manuscript_figures/}
        {github.com/chuckpr/BvP\_manuscript\_figures/}
\end{itemize}
\section*{Journal Articles in Preparation}
\begin{itemize}
    \item Pepe-Ranney C$^{*}$ and Campbell A$^{*}$, Koechli C, Berthrong S, 
        Buckley DH. \textbf{Charting the flow of carbon through a soil
            microbial community with high resolution DNA stable isotope
            probing.}\\
        {\small$^{*}$co-first authors}\\
        Code for manuscript figures can be found here:\\
        \href{http://nbviewer.ipython.org/github/chuckpr/CSIP_succession_data_analysis}
        {nbviewer.ipython.org/github/chuckpr/CSIP\_succession\_data\_analysis}
    \item Pepe-Ranney C, Campbell A, Buckley DH. 
        \textbf{Community genomics of soil cellulose degraders discovered by nucleic acid stable isotope probing}\\
        Code for manuscript figures can be found here:\\
        \href{http://nbviewer.ipython.org/github/chuckpr/CG-SIP}
        {nbviewer.ipython.org/github/chuckpr/CG-SIP/tree/master/}
    \item Koechli C, Pepe-Ranney C, Campbell A, Buckley DH. \textbf{Mapping
        carbon flow through both 16S rRNA and 16S rRNA genes from an
        agricultural soil using stable isotope probing provides
        insights into bacterial metabolism.}
    \item Hahn C, Hall EK, Pepe-Ranney C, Oyler-McCance S.
        \textbf{Evaluating the gut and cloacal bacterial community of
        cowbirds: a potential mechanism for enhanced immunity.}\\
        Code for sequence analysis and manuscript figures can be found here:\\
        \href{http://nbviewer.ipython.org/github/chuckpr/cowbird}
        {nbviewer.ipython.org/github/chuckpr/cowbird}
\end{itemize}
\section*{Invited Talks}
\begin{itemize}
    \item $^{14}$ C and microbial diversity study of Yellowstone siliceous
        stromatolites: searching for the depositional community.  2009.
        Microbiology Supergroup, University of Colorado - Boulder.
    \item Cyanobacterial construction of finely laminated siliceous
        stromatolites in a Yellowstone National Park hot spring. 2012.
        Astrobiology Science Conference - Microbes in Lithifying
        Systems.
\end{itemize}
%\section*{Selected Conference Abstracts}
%\begin{itemize}
%    \item Pepe-Ranney C, Berelson WM, Corsetti FA, Spear JR. Microbial
%        Diversity of a modern stromatolite analog in Yellowstone
%        National Park, Wyoming: Searching for the depositional
%        community. International Symposium on Microbial Ecology 13,
%        2010.  
%    \item Pepe-Ranney C, Berelson WM, Corsetti FA, Spear JR.  Microbial
%        diversity of a living stromatolite in Yellowstone National Park,
%        Wyoming: Learning how a stromatolite grows. American Geophysical
%        Union Fall Meeting, 2010.
%    \item Osburn M, Sessions A, Pepe-Ranney C, Spear JR. Comparison of
%        lipidomics and genomics to describe hydrothermal communities in
%        Yellowstone National Park.  American Geophysical Union Fall
%        Meeting, 2010.
%    \item Corsetti FA, Berelson WM, Spear JR, Pepe-Ranney C, Beaumont W,
%        Mata S. The effects of spring level on stromatolite lamination
%        and morphology, Yellowstone National Park.  Geological Society
%        of America 2011.
%    \item ASM Boston
%    \item DOE meetings
%    \item ASM Philidelphia
%\end{itemize}

%\bigskip
%
% Footer
\begin{center}
  \begin{footnotesize}
    Last updated: \today \\
  \end{footnotesize}
\end{center}

\end{document}

% vim: set paste:
