% LaTeX Curriculum Vitae Template
%
% Copyright (C) 2004-2009 Jason Blevins <jrblevin@sdf.lonestar.org>
% http://jblevins.org/projects/cv-template/
%
% You may use use this document as a template to create your own CV
% and you may redistribute the source code freely. No attribution is
% required in any resulting documents. I do ask that you please leave
% this notice and the above URL in the source code if you choose to
% redistribute this file.

\documentclass[letterpaper]{article}

\usepackage{hyperref}
\usepackage{geometry}

% Comment the following lines to use the default Computer Modern font
% instead of the Palatino font provided by the mathpazo package.
% Remove the 'osf' bit if you don't like the old style figures.
\usepackage[T1]{fontenc}
\usepackage[sc,osf]{mathpazo}

% Set your name here
\def\name{\textbf{Charles Pepe-Ranney}}

% Replace this with a link to your CV if you like, or set it empty
% (as in \def\footerlink{}) to remove the link in the footer:
\def\footerlink{}

% The following metadata will show up in the PDF properties
\hypersetup{
  colorlinks = true,
  urlcolor = blue,
  pdfauthor = {\name},
  pdfkeywords = {},
  pdftitle = {\name: Curriculum Vitae},
  pdfsubject = {Curriculum Vitae},
  pdfpagemode = UseNone
}

\geometry{
  body={7.0in, 9.5in},
  left=0.75in,
  top=0.60in,
  bottom=0.60in
}

% Customize page headers
\pagestyle{myheadings}
\markright{\name}
\thispagestyle{empty}

% Custom section fonts
\usepackage{sectsty}
\sectionfont{\rmfamily\mdseries\itshape\large}
\subsectionfont{\rmfamily\mdseries\itshape\normalsize}

% Other possible font commands include:
% \ttfamily for teletype,
% \sffamily for sans serif,
% \bfseries for bold,
% \scshape for small caps,
% \normalsize, \large, \Large, \LARGE sizes.

% Don't indent paragraphs.
\setlength\parindent{0em}

% Make lists without bullets
\renewenvironment{itemize}{
  \begin{list}{}{
    \setlength{\leftmargin}{1.5em}
  }
}{
  \end{list}
}

% include multibib package
%\usepackage{multibib}

\begin{document}

% Place name at left
{\Large \name}

% Alternatively, print name centered and bold:
%\centerline{\large \bf \name}

\vspace{0.125in}

\begin{minipage}{0.45\linewidth}
  Microbial Genomics Data Scientist \\
  AgBiome \\
  Research Triangle Park, NC 27709
\end{minipage}
\begin{minipage}{0.45\linewidth}
  \begin{tabular}{ll}
    Phone: & (575) 313-0993 \\
    Email: & \href{mailto:cpeperanney@agbiome.com}{\tt cpeperanney@agbiome.com} \\
    Website: & \href{http://chuckpr.github.io/blog}{\tt chuckpr.github.io/blog} 
  \end{tabular}
\end{minipage}
\section*{Professional Preparation}
\begin{itemize}
    \item \textbf{B.S. Engineering (high honors)} - Environmental Science Specialty, Colorado School of Mines 2006.
    \item \textbf{M.S. Environmental Engineering} - Biotechnology and Environmental Microbiology Emphasis, Colorado School of Mines 2009.
    \item \textbf{PhD Environmental Science and Engineering Division}, Colorado School of Mines 2012.
\end{itemize}
\section*{Relevant Experience}
\begin{itemize}
{\small
    \item Methods in microbiome analysis teaching fellow at
        Microbial Diversity Course, 2010-2013 (Marine Biology Laboratory,
        Woods Hole, MA).
    \item Proficient with \textbf{Python}, \textbf{R}, JavaScript,
        Latex, Bash, Amazon AWS, Docker, and Linux system
        administration.
    \item Thorough understanding and fluent with many data
        science tools in Python, R, and JavaScript including Jupyter,
        \\ ggplot2 (R), Plotly, scikit-learn (Python), Matplotlib (Python), Pandas (Python), phyloseq (R), plyR/dplyR/tidyR (R), Bokeh
        (Python), D3.js
}
\end{itemize}
%\section*{Synergistic Activities}
%\begin{itemize}
%    \item \emph{Teaching fellow, Microbial Diversity Course.}  The summers
%        of 2010, 2011 and 2012 I taught bioinformatics methods and
%        molecular laboratory techniques as a Microbial Diversity Course
%        teaching fellow (Marine Biology Laboratory, Woods Hole, MA).
%        Additionally, I set-up and installed all computational analysis
%        software on the Linux machines and served as the system
%        administrator.  I will be returning to the course in the same
%        capacity for the 2013 summer.  Students are eager to learn a
%        variety of computational methods and possess a wide range of
%        prior computational experience.  Students leave with basic
%        foundational skills to manage the flood of biological sequence
%        information typical of contemporary microbiology investigations.
%\end{itemize}

\section*{Appointments}
\begin{itemize}
    \item \textbf{Research Assistant}, Environmental Science and Engineering
        Division, Colorado School of Mines (2006-2012)
    \item \textbf{Postdoctoral Researcher}, Department of Crop and Soil
        Sciences, Cornell University (2013-2015)
    \item \textbf{Research Associate}, School of Integrative Plant Science,
         Cornell University (2015-2016)
    \item \textbf{Teaching Fellow - Microbial
        Diversity Course}, Marine Biological Laboratory (Woods Hole, MA) (2010-2013)
\end{itemize}
\section*{Awards and Fellowships}
\begin{itemize}
    \item \textbf{2015} Poster Prize, AEM Gordon Research Seminar
    \item \textbf{2010, 2011, 2012 and 2013 Teaching fellow for the Microbial
        Diversity Course} at the Marine Biological Laboratory, Woods
        Hole.  Course Directors: Daniel Buckley and Steve Zinder.
    \item \textbf{2006 Outstanding Graduating Senior Award}, Colorado School of Mines - Environmental Science and Engineering Division
    \item \textbf{2005 and 2006 Department of Energy Science Undergraduate Laboratory Internship
        (SULI)} at Idaho National Lab
    \item \textbf{2006 Idaho National Lab Undergraduate Scholarship}
\end{itemize}
\section*{Publications in Refereed Journals}
\begin{itemize}

\item Choudoir MJ, Pepe-Ranney C, Buckley DH.
    \textbf{Diversification of secondary metabolite biosynthetic gene clusters
    coincides with lineage divergence in \textit{Streptomyces.}} 2018.
    \textit{Antibiotics} 7(1), 12

\item Jackson EW, Pepe-Ranney C, Debenport SJ, Buckley DH, Hewson I.
    \textbf{The microbial landscape of sea stars and the anatomical and
    interspecies variability of their microbiome.} 2018.
    \textit{Frontiers in Microbiology} 9, 1829

\item Whitman T, Pepe-Ranney C, Enders A, Koechli C,
    Campbell A, Buckley DH, Lehmann J. \textbf{Dynamics of microbial community
    composition and soil organic carbon mineralization in soil following addition
    of pyrogenic and fresh organic matter.} 2016. \textit{ISMEJ} 10(12), 2918

\item Pepe-Ranney C, Campbell AN, Koechli CN, Berthrong S, Buckley DH.
    \textbf{Unearthing the ecology of soil microorganisms using a high resolution
    DNA-SIP approach to explore cellulose and xylose metabolism in soil.} 2016.
    \textit{ Frontiers in Microbiology} 7, 703

\item Pepe-Ranney C, Berelson WM, Corsetti FA, Treants M, Spear JR.
    \textbf{Cyanobacterial construction of hot spring siliceous stromatolites in
    Yellowstone National Park, Wyoming}, 2012, 
    \textit{Environmental Microbiology} 14(5), 1182-1197. 

\item Pepe-Ranney C, Koechli C, Potrafka R, Garcia-Pichel F, Andam C, Eggleston E, Buckley DH. 
    \textbf{Non-cyanobacterial diazotrophs mediate dinitrogen fixation in
    biological soil crusts during early crust formation.}, 2015, \textit{ISMEJ}

\item Pepe-Ranney C and Hall EK. \textbf{The effect of carbon subsidies on
    planktonic niche partitioning and recruitment during biofilm
    assembly.}, 2015, \textit{Frontiers in Microbiology}, 6:703. 

\item Berelson WM, Corsetti FA, Pepe-Ranney C, Hammond DE, Beaumont W, Spear
    JR. \textbf{Hot spring siliceous stromatolites in Yellowstone National Park:
    assessing growth rates and laminae formation}, 2011, {\it Geobiology}
    9(5), 411-424.

\item Osburn MR, Sessions AL, Pepe-Ranney C, Spear JR. \textbf{Hydrogen-isotopic
    variability in fatty acids from Yellowstone National Park hot spring
    microbial communities}, 2011, {\it Geochimica et Cosmochimica Acta}
    75(17), 4830-4845.

\item Br\"{a}uer S, Vuono D, Carmichael M, Pepe-Ranney C, Strom A, Rabinowitz
    E, Buckley DH, Zinder S. \textbf{Microbial sequencing analyses suggest
    the presence of a fecal veneer on indoor climbing wall holds.}, 2014, 
    {\it Current Microbiology} 69(5), 681-689.

\item Wallace B, Roberts A, Pollet R, Venkatesh M, 
    Guthrie L, O'Neal S, Ingle J, Robinson S, Dollinger M, Figueroa E, 
    McShane S, Jin J, Frye S, Zamboni W, Pepe-Ranney C, Mani S, Kelly
    L, and Redinbo M. \textbf{Structure and Inhibition of Firmicutes
    Bacterial b-Glucuronidases to Alleviate Drug-Induced GI Toxicity}
    \textit{Chemistry \& Biology}\\

\end{itemize}
\section*{Invited Talks}
\begin{itemize}
    \item 50,000 genomes and counting: how to manage and explore the data from
        your giant collection of microbial isolates. 2018.
        New Frontiers in Plant Biology Workshop CBGP-Madrid
    \item Leveraging analytics: putting big data to Good use. 2016.
        Ag Biotech Professional Forum at the NC Biotechnology Center.
    \item Tracking carbon into and through the soil microbial community with
        DNA-SIP. 2016. EcoFAB Workshop, Joint Genome Institute
    \item Targeting unknowns just underfoot: microbial ecology and community
        genomics of C cycling in soil informed and enabled with DNA-SIP. 2015.
        AGU Fall Meeting - \textit{Understanding Microbial Processes, Dependencies, and
        Impacts through 'omics} session
    \item
        Cyanobacterial construction of finely laminated siliceous
        stromatolites in a Yellowstone National Park hot spring. 2012.
        Astrobiology Science Conference - \textit{Microbes in Lithifying
        Systems.}
    \item $^{14}$ C and microbial diversity study of Yellowstone siliceous
        stromatolites: searching for the depositional community. 2009.
        Microbiology Supergroup, University of Colorado - Boulder.
\end{itemize}
%\section*{Selected Conference Abstracts}
%\begin{itemize}
%    \item Pepe-Ranney C, Berelson WM, Corsetti FA, Spear JR. Microbial
%        Diversity of a modern stromatolite analog in Yellowstone
%        National Park, Wyoming: Searching for the depositional
%        community. International Symposium on Microbial Ecology 13,
%        2010.  
%    \item Pepe-Ranney C, Berelson WM, Corsetti FA, Spear JR.  Microbial
%        diversity of a living stromatolite in Yellowstone National Park,
%        Wyoming: Learning how a stromatolite grows. American Geophysical
%        Union Fall Meeting, 2010.
%    \item Osburn M, Sessions A, Pepe-Ranney C, Spear JR. Comparison of
%        lipidomics and genomics to describe hydrothermal communities in
%        Yellowstone National Park.  American Geophysical Union Fall
%        Meeting, 2010.
%    \item Corsetti FA, Berelson WM, Spear JR, Pepe-Ranney C, Beaumont W,
%        Mata S. The effects of spring level on stromatolite lamination
%        and morphology, Yellowstone National Park.  Geological Society
%        of America 2011.
%    \item ASM Boston
%    \item DOE meetings
%    \item ASM Philidelphia
%\end{itemize}

%\bigskip
%
% Footer
\begin{center}
  \begin{footnotesize}
    Last updated: \today \\
  \end{footnotesize}
\end{center}

\end{document}

% vim: set paste:
